\chapter{Introduction}

The nucleons (protons and neutrons) are spin-half fermions, which are bound states of QCD.  Exactly how the quarks and gluons dynamically combine to produce the net spin-half of the nucleon is not clear.  Striking results of $g_1$ and $g_2$ measurements performed by the European Muon Collaboration (EMC) in 1988 \cite{pdfs-leader:1988} demonstrated that only $30\%$ of the net spin of the nucleons can be attributed to the quark spin.  This result became known was the \textit{proton spin crisis}, and remains largely unresolved.  The understanding of quark orbital angular momentum in the nucleon, and it's contribution to the net spin, is now of vital importance.  \\

Addressing the question of orbital angular momentum distributions of partons within nucleons necessitates moving beyond a co-linear picture of parton interactions.   During the early 1990s, theoretical tools began to emerge that are now being used to study quark dynamics in three-dimensions.  Transverse momentum dependent functions (TMDs) naturally extend the co-linear parton distribution functions (PDFs) to include intrinsic quark momentum in the plane transverse to the hard probe \cite{tmds-mulders:1995, tmds-bacchetta:2006}.