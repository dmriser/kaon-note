\chapter{Introduction}

% Setting up the problem (where is the proton spin coming from)
The nucleons (protons and neutrons) are spin-half fermions, which are bound states of QCD.  Exactly how the quarks and gluons dynamically combine to produce the net spin-half of the nucleon is not clear.  Striking results of $g_1$ and $g_2$ measurements performed by the European Muon Collaboration (EMC) in 1988 \cite{pdfs-leader:1988} demonstrated that only $30\%$ of the spin of the nucleons can be attributed to quark spin.  This result became known as the \textit{proton spin crisis}, and remains largely unresolved.  The understanding of quark orbital angular momentum in the nucleon, and it's contribution to the proton spin, is now of vital importance.  \\

% Introducing important character (TMD)
Addressing the question of orbital angular momentum distributions of partons within nucleons necessitates moving beyond a co-linear picture of parton interactions.   During the early 1990s, theoretical tools began to emerge that are now being used to study quark dynamics in three-dimensions.  Transverse momentum dependent functions (TMDs) naturally extend the co-linear parton distribution functions (PDFs) to include intrinsic quark momentum in the plane transverse to the hard probe \cite{tmds-mulders:1995, tmds-bacchetta:2006}.  \\

% Introducing important character (SSAs)
Sadly, TMDs are not directly observable.  Despite this fact, single spin asymmetry (SSA) measurements of semi-inclusive deeply inelastic scattering (SIDIS) have proved useful in recent years as inputs for phenomenological extraction of TMD parton distribution functions (TMD PDFs, sometimes just called TMDs) and TMD fragmentation functions (TMD FFs or simple FFs) \cite{tmds-airapetian:2009, tmds-airapetian:2012, tmds-aghasyan:2017}.  Because of the absence of a TMD PDF, semi-inclusive annihilation of $e^+ e^- \rightarrow h_1 h_2 X$ has proved useful for accesing the TMD FFs (add a reference the Prokudin's extraction of Collins here).  