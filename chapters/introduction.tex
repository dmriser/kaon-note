\chapter{Introduction}

Measurement of parton distribution functions (PDFs) have been performed at experiments around the world (HERMES, COMPASS, and Jefferson Lab) decades.  The PDFs (at leading order) describe the probability to observe a quark with a given momentum fraction $x$ in a hadron.  There is one such function for each quark flavor in each hadron.  The unpolarized PDF is now known quite well, and the polarized PDFs have also been measured.  During the measurement of polarized PDFs in 1989 the European Muon Collaboration (EMC) observed that only 30\% of the total spin of the proton appeared to be due to the spin of the quarks.  This result came to be known as the proton spin crisis.  

One possible resolution to this problem is that the quarks carry orbital angular momentum inside of the proton and this contributes to the total observed spin.  In this case, measurements of the three-dimensional momentum structure of the quarks inside of hadrons is expected to be very useful.  The transverse momentum dependent parton distribution functions (TMD PDFs) describe the quark momenta in both the longitudinal direction $x$ (defined by the hard momentum transfer direction) and the momentum in the plane transverse to that as well $\mathbf{p_T}$.  



