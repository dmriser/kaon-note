\chapter{Experiment}

\section{CEBAF at Jefferson Lab}
This section describes briefly the accelerator facility. 

% Accelerator overview 
% Injector description 
% Generation of beam polarization (Pockells cell)
% Linear accelerator description 
% Delivery to hall B 

This work was performed at Jefferson National Lab, making use of the continuous electron beam accelerator facility (CEBAF).  

\section{The CLAS Detector}
This section describes the sub-systems of CLAS and how they work together.  This section should also mention data aquisition.

% Major technical specifications overview
% List of major sub-detectors 
% Description of the role of each sub-detector 
% Description of Data Aquisition hardware and software, data format 
% Description of Track Reconstruction

\section{Helicity Determination}
During the course of the E1-F run period the beam helicity convention was changed by the insertion of a half-wave plate at the injector.  Our definition of $+$ and $-$ helicity must change in accordance with these wave-plate insertions.  To monitor these changes, the value of $A_{LU}^{\sin\phi}$ for $\pi^+$ is recorded for every run.  Whenever the asymmetry (which has a magnitude of around $3\%$) changes sign, we know that the sign convention has changed.  These changes are then taken into account in the data analysis.   

\easyFigure{image/plots/basic-analysis/bsa-waveplate.pdf}{The waveplate position is determined and corrected by plotting the BSA for $\pi^{+}$ mesons as a function of the run.  The top panel shows the corrected results, the bottom shows the results before changing the helicity.}

\section{Determination of Good Run List}
The total dataset contains 831 runs.  Due to the complex experimental setup, it is not uncommon for run conditions to change during a few of the runs such that the data collected are not of analysis quality.  Imagine as a simple example that the liquid hydrogen target boils, the density is suddenly decreased, and the number of recorded events drops drastically (but the Faraday Cup charge would look the same).  For this reason, a good run list is constructed and used in the analysis. \\

To construct this list, we simply count good electrons in every file and normalize that by the accumulated charge for that file.  While the number of events collected varies from run to run the ratio defined above is a stable quantity -- provided that the run conditions do not vary greatly.  Runs which are within 3 standard deviations of the mean (calculated over the dataset) are used as good runs.  The good run list used for this analysis contains 522 runs.  

\easyFigure{image/plots/basic-analysis/inclusive-rates.png}{Inclusive electrons per file normalized by the total charge accumulated for the file.  This quantity is used to make a good run list.}
